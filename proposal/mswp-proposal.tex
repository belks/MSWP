\documentclass[a4paper,11pt]{article}

\usepackage{graphicx}
\usepackage{natbib}
\usepackage[utf8]{inputenc}
\usepackage{tabularx}
\usepackage{hyperref}
\usepackage{color}
\usepackage[usenames,dvipsnames,svgnames,table]{xcolor}
% \usepackage{mathptmx} % Times New Roman

\setlength{\topmargin}{-0.4mm} % (1in=25.4mm)-0.4mm=25mm
\setlength{\textheight}{243.119mm} % 297mm-40mm-10mm-(11pt=3.881mm)=
\setlength{\oddsidemargin}{-0.4mm} % (1in=25.4mm)-0.4mm=25mm
\setlength{\textwidth}{160mm} % 210mm-50mm=160mm
\setlength{\headheight}{0mm}
\setlength{\headsep}{0mm}
\setlength{\footskip}{15mm}

\providecommand*{\note}[1]{\small \textcolor{RoyalBlue}{\begin{minipage}{\textwidth}{#1}\end{minipage}}}

% --------------------------------------------------------------

\providecommand*{\ShortTitle}{WaveMeIn}
\providecommand*{\FullTitle}{WaveMeIn: Authentication via Brain Waves}

% --------------------------------------------------------------

\title{\textbf{\sffamily\Huge \ShortTitle}\\ 
{\textbf{\sffamily\Large \FullTitle}}
\vspace{1cm}}

\author{
{\em 188.407: Management von Software Projekten} \vspace{1cm} \\
Group: 10\bigskip \\
Belk Stefan \\ {\small 0750926, 937, \href{mailto:belk.stefan@gmail.com}{belk.stefan@gmail.com}}\\
Petz Thomas \\ {\small 0601280, 937, \href{mailto:e0601280@student.tuwien.ac.at}{e0601280@student.tuwien.ac.at}}\\
Causevic Alma \\ {\small 0847805, 534, \href{mailto:alma.causevic@hotmail.com}{alma.causevic@hotmail.com}}\\ 
Causevic Amra  \\ {\small 0649241, 534, \href{mailto:amra.causevic@hotmail.com}{amra.causevic@hotmail.com}}\\ 
Seebacher David \\ {\small 0327243, 534, \href{mailto:david.seebacher@student.tuwien.ac.at}{david.seebacher@student.tuwien.ac.at}}\\
\vspace{4cm}
}

\begin{document}

\begin{titlepage}
\maketitle

\end{titlepage}

% --------------------------------------------------------------

\thispagestyle{empty}
\tableofcontents
\pagebreak

\setcounter{page}{1}


% --------------------------------------------------------------

\note{
\textbf{Formal constraints}
\begin{itemize}
\item	  Font: Times New Roman oder Computer Modern (\LaTeX default)
\item    Fontsize: 11pt
\item     Single line spacing
\item     Margins: 2.5cm side and top/bottom
\item     \fbox{Language: ENGLISH}
\item    The proposal template should be filled incrementally. I.e., at the end there should be a full project proposal in a single PDF file.
\end{itemize}
\textbf{Available templates}
\begin{itemize}
\item     Proposal (mswp-proposal.tex)
\item     Costs (costs.xls, costs.ods)
\end{itemize}
\textbf{Supplemental material}
\begin{itemize}
\item     FWF salary scheme (\href{http://www.fwf.ac.at/de/projects/personalkostensaetze.html}{http://www.fwf.ac.at/de/projects/personalkostensaetze.html})
\item     Travel cost regulation (\href{http://www.fwf.ac.at/de/faq/reisegebuehrenvorschrift.html}{http://www.fwf.ac.at/de/faq/reisegebuehrenvorschrift.html})
\item     Ethical issues form (ethical-issues.rtf)
\end{itemize}
}
\pagebreak

% --------------------------------------------------------------
\section{Synopsis}
\label{sect:synopsis}
\subsection{Project Idea}
WaveMeIn is a research project to create a new type of secure login mechanism. It consists of a small device worn by the user at the ear which authenticates the user based on brain waves.

\subsection{Why do we need it?}
At the time of this proposal the most used ways for authentication are manually typed passwords or biometric authentication methods. However all of the previous methods have some security problems or are simply not user-friendly. Typed passwords are easy to spy out simply by looking at the keyboard of the user or the traces of the fingers on touch displays. In the case of biometric authentication, there are for example face recognition, iris or fingerprint scans. Face recognition software can easily be tricked by face masks or photographs and moreover depends on good light conditions, the quality of the images of the web camera and other factors. Fingerprint and iris scans are the most secure options of the authentication methods mentioned before. However they also have many disadvantages. Iris scans are not practical since the hardware required cannot easily be integrated into small devices and it is not user-friendly to require the user to place his eye very close to the scanner every time he/she wants to unlock a device. Fingerprint sensors are known to fail to recognize the fingerprint correctly quite often and it is also a not very user-friendly authentication method for handicapped people that may not reach the sensor or may not have any fingers at all. 

\subsection{How does it work?}
Brain waves are a secure and user-friendly alternative authentication method. The idea is to create a small device, called Wavy, that can be worn at the ear of the user in the same style as bluetooth headsets are already worn for communication today. The Wavy measures the brain waves near the ear in case a login is required by a client device that is connected via bluetooth. It listens for a brain wave pattern that was previously trained by the user as a password. If the correct pattern was detected by the Wavy it transmits a OK signal back to the client device. 

\subsection{Why should somebody care?}
Nowadays people are forced to type their passwords in public places which is a security risk and also not a very efficient way for authentication. Especially when typing in password on small devices such as mobile phones this authentication method is also very error prone due to the small keyboard interfaces. On the one side people are lazy and do not want to remember and enter long and complicated passwords, but on the other side they are also concerned about the security of their data and their privacy. So the users are in need of a more secure and easier way of authentication.

\subsection{Who are the beneficiaries of the results?}
Basically everybody can benefit from the WaveMeIn project since it is usable in the daily life. Especially for handicapped people it is a new and more easy to use option to log into their devices. Also it grants a higher level of security than existing authentication methods so it is also well suited for environments where higher security is needed, such as access authentication is modern research labs and government or military facilities.

For our product to succeed, we need to invest into research in the area of brain wave detection and analysis. This investment can improve our understanding of this topic. After a commercial success, we have to enhance our product. This means we have to invest further into brain wave research. On the other side, we can make our world more secure. It makes hacking of accounts and password fraud more complicated.

\subsection{Problem classification}
The task of detecting brain waves it tightly connected to the research areas of Neuroscience, Pattern Recognition and Machine Learning. In the field of Neuroscience it touches the areas of not invasive brain computer interfaces and neural oscillation. Since detecting and reliably identifying brain waves at the location near the ears is still technically immature the project can be seen as basic research in this area. The following research questions have to be answered before a prototype can be developed.
\begin{itemize}
	\item Detecting braves at the ears
	\item Recognize brain wave patters
	\item Distinguish correct patterns from random signals
	\item Distinguish brain waves from different users
\end{itemize}
On the other hand if we take the Wavy into account, which should be the resulting product, this project is also an applied research project. It further touches the fields of computer security and privacy.


% --------------------------------------------------------------
\section{Introduction and problem description}
\label{sect:intro}

\note{
\begin{itemize}
	\item {\em Length: 2-3 pages}
	\item {\bf Why?}
	\item Introduction
	\item Context
	\item What is the current situation?
	\item What is the open/unresolved problem or opportunity?
	\item Why is it a problem?
	\item What is unknown?
	\item What could be improved?
	\item Explanation of fundamental terms and basic definitions.
\end{itemize}
}

% --------------------------------------------------------------
\section{Project goals and deliverables}
\label{sect:goals}

\note{
\begin{itemize}
	\item {\em Length: 1-2 pages}
	\item What is the goal of the project?
	\item Research questions
	\begin{itemize}
		\item 	    What are the hypotheses that are to be investigated?
		\item 	    Main hypothesis \& sub hypotheses
	\end{itemize}
	\item Which results should be achieved with the project?
	\begin{itemize}
	    	\item 	   What will be known afterwards that is not known now?
		\item	    What will be created that does not exist now?
	\end{itemize}
	\item Non-goals (What will not be part of the project? What will not be done?)
\end{itemize}
}

% --------------------------------------------------------------
\section{Scientific relevance and innovative aspects}
\label{sect:relevance}

\note{
\begin{itemize}
\item {\em Length: 1-2 pages}
\item Why is the project scientifically interesting?
\item Did others point out that this is an open question?
\item What are the innovative aspects that make it interesting?
\item How could the project break new ground scientifically?
\item To what extent are the objectives ambitious and beyond the state of the art (e.g. novel concepts and approaches or development across disciplines)?
\end{itemize}
}

% --------------------------------------------------------------
\section{State of the art / current knowledge}
\label{sect:star}

\note{
\begin{itemize}
\item {\em Length: 2-5 pages}
\item What results and approaches have already been presented in this or related areas?
\item Relation to the international scientific work in the field (international status of the research)
\item Description and critical discussion of related scientific work
\end{itemize}
}

% --------------------------------------------------------------
\section{Method}
\label{sect:method}

\note{
\begin{itemize}
\item {\em Length: 2-5 pages}
\item {\bf How?}
\item How should the expected results be achieved?
\item What method(s) will be applied? (e.g., empirical study, user-centered design, prototype implementation,...)
\item Description of the methods.
\item Justifications for chosen methods.
\end{itemize}
}

% --------------------------------------------------------------
\section{Detailed description of the workpackages}
\label{sect:workplan}

\note{
\begin{itemize}
\item {\em Length: 2-4 pages}
\item Structuring the project into self-contained parts.
\item Additional verbal descriptions.
\item Work packages
    \begin{itemize}
    \item title
    \item goal(s)
    \item description
    \item expected results
    \item responsible person(s)
    \item dependencies
    \end{itemize}
\end{itemize}
}

% --------------------------------------------------------------
\section{Time plan (Gantt chart)}
\label{sect:timeplan}

\note{
\begin{itemize}
\item {\em Length: 1-2 pages}
\item Realistic estimation of schedule based on workpackages.
\item Including milestones (not only when but also what is to be achieved for each milestone).
\item Generation of a Gantt chart. (Including phases, milestones, buffer times, critical areas, etc.)
\end{itemize}
}

% --------------------------------------------------------------
\section{Human resources / team}
\label{sect:team}

\note{
\begin{itemize}
\item {\em Length: 1-2 pages}
\item Description of the team that is needed to carry out the project. (For the execution phase of the project, not the planning phase.)
\item How many people?
\item To what extent are individual members needed?
\item What knowledge, skills, and experiences are needed for each member?
\item Demonstrate that the members will be able to carry out the project successfully.
\item Work structure
	\begin{itemize}
	\item     Who will lead the project?
	\item     How do they work together?
	\item     Management and coordination
		\begin{itemize}
		\item 	        What communication structures will be established? (e.g., mailing list, blog, CMS, CVS, ...)
		\item 	        How often will meetings take place? (Who will participate?)
		\item 	        How will the work be documented?
		\item 	        How will information be stored and shared?
		\end{itemize}
	\end{itemize}
\item Cooperations
	\begin{itemize}
	\item     Will external cooperators be part of the project? (e.g., other research institutions or companies)
	\item     What is their role?
	 \item    Why are they needed?
	\end{itemize}
\end{itemize}
}

% --------------------------------------------------------------
\section{Costs}
\label{sect:costs}

\note{
\begin{itemize}
\item {\em Length: 2-3 pages}
\item Rough estimation of cost in form of calculation (table(s)) + descriptive text.
\item Justification for the personnel and non-personnel costs (equipment, material, travel and other costs)
\item An Excel template is provided as supplementary material to support budgeting.
\item Personnel costs
	\begin{itemize}
	\item     Justification for the personnel to be assigned to the project (type of position(s), description of nature of work, length and extent of involvement in the project)
	\item     The application should include all persons who will be required for the proposed project (project lead, researchers, developers, advisory board, etc.). The available legal categories of employment are contracts of employment for full- or part-time employees (DV) and reimbursement for work on an hourly basis (GB). In addition, a part-time contract of employment (DV 50\%, ``studentische Mitarbeiter'') may be requested for people who have not yet completed a Master or Diploma program (Diplom) in the relevant subject.
	 \item    The justification of the requested personnel should contain:
		\begin{itemize}
		\item 	        description of type of work;
		\item 		        extent of involvement (part-time contracts are permitted).
		\end{itemize}
	\item Exact numbers of employment categories can be found on the FWF Website (\href{http://www.fwf.ac.at/de/projects/personalkostensaetze.html}{http://www.fwf.ac.at/de/projects/personalkostensaetze.html})
	\end{itemize}
\item Equipment costs
	\begin{itemize}
	\item     Indicate reasons for equipment costs. The ``scientific equipment'' category includes instruments, system components, costs for the use of software required by the project and other durable goods provided the cost per item (including VAT) exceeds EUR 1,500.00.
	\end{itemize}
\item Material costs
	\begin{itemize}
	\item     This category encompasses consumables and smaller pieces of equipment where the cost per item is below EUR 1,500.00 including VAT. The calculation of requested material costs should be justified with reference to the schedule, work plan and experimental plan. Experience with previous projects should be taken into account.
	\end{itemize}
\item Travel costs
	\begin{itemize}
	\item     Funding may be requested for the costs of project-specific travel and accommodation, field work, expeditions, etc. Applicants are to provide a detailed travel (cost) plan broken down by project participant. For brief stays, the calculation of the travel and accommodation costs should be based on the federal regulations governing travel costs (RGV). The RGV rates governing Austria and abroad may be found in the FAQs on the FWF Website (\href{http://www.fwf.ac.at/de/faq/reisegebuehrenvorschrift.html}{http://www.fwf.ac.at/de/faq/reisegebuehrenvorschrift.html}). For longer stays an appropriate and comprehensible cost plan should be prepared.
	\end{itemize}
\item Other costs
	\begin{itemize}
	\item     Independent contracts for work and services (costs for work of clearly defined scope and content assigned to individuals, provided that this is scientifically justifiable and economical)
    	\item     Costs that cannot be included under personnel, equipment, material or travel costs, such as:
		\begin{itemize}
		\item         reimbursement of costs towards or for the use of research facilities, e.g. of large-scale research facilities (project-specific 'equipment time'). Applicants should obtain and submit multiple offers;
		\item         costs for project-specific work carried out outside the applicant's research institution (e.g. for analysis work performed elsewhere, for interviews, for sample collection, for preparation of thin slices etc.). Applicants should obtain and submit multiple offers;
		\item         honoraria for test persons;
		\end{itemize}
	\end{itemize}
\end{itemize}
}

% --------------------------------------------------------------
\section{Expected implications and risks}
\label{sect:implication-risk}

\note{
\begin{itemize}
\item {\em Length: 1-2 pages}
\item Importance of the expected results for the discipline
	\begin{itemize}
	\item     To what extent does the proposed research address important challenges?
	\end{itemize}
\item Importance of the expected results for other areas
\item What are possible risks of the project and how can they be alleviated?
	\begin{itemize}
	\item     What factors could lead to a failure of the project?
	\item     Which factors or persons could support the project and increase the chance for success?
	\item     What if important team members leave the project?
	\end{itemize}
\end{itemize}
}

% --------------------------------------------------------------
\section{Ethical considerations \& security issues}
\label{sect:ethics-security}

\note{
\begin{itemize}
\item {\em Length: 1-2 pages}
\item Provide a brief explanation of the ethical issue involved and how it will be dealt with appropriately.
\item Are there any security-sensitive issues that apply to your proposal?
\end{itemize}
}

% --------------------------------------------------------------
% APPENDIX
\begin{appendix}

\pagebreak

% --------------------------------------------------------------
% References
\phantomsection
\addcontentsline{toc}{section}{References}

\bibliographystyle{apalike}
\bibliography{mswp2013-proposal}

\pagebreak

% --------------------------------------------------------------
% Abbreviations
\section*{Abbreviations}
 \addcontentsline{toc}{section}{Abbreviations}
 
 \begin{description}
  \item[MSWP] Management von Software Projekten
  \item[WP] Work Package
 \end{description}

\end{appendix}


\end{document}
