\documentclass[a4paper,11pt]{article}

\usepackage{graphicx}
\usepackage{natbib}
\usepackage[utf8]{inputenc}
\usepackage{tabularx}
\usepackage{hyperref}
\usepackage{color}
\usepackage[usenames,dvipsnames,svgnames,table]{xcolor}
% \usepackage{mathptmx} % Times New Roman

\setlength{\topmargin}{-0.4mm} % (1in=25.4mm)-0.4mm=25mm
\setlength{\textheight}{243.119mm} % 297mm-40mm-10mm-(11pt=3.881mm)=
\setlength{\oddsidemargin}{-0.4mm} % (1in=25.4mm)-0.4mm=25mm
\setlength{\textwidth}{160mm} % 210mm-50mm=160mm
\setlength{\headheight}{0mm}
\setlength{\headsep}{0mm}
\setlength{\footskip}{15mm}

\providecommand*{\note}[1]{\small \textcolor{RoyalBlue}{\begin{minipage}{\textwidth}{#1}\end{minipage}}}

% --------------------------------------------------------------

\providecommand*{\ShortTitle}{WaveMeIn}
\providecommand*{\FullTitle}{WaveMeIn: Authentication via Brain Waves}

% --------------------------------------------------------------

\title{\textbf{\sffamily\Huge \ShortTitle}\\ 
{\textbf{\sffamily\Large \FullTitle}}
\vspace{1cm}}

\author{
{\em 188.407: Management von Software Projekten} \vspace{1cm} \\
Group: 10\bigskip \\
Belk Stefan \\ {\small 0750926, 937, \href{mailto:belk.stefan@gmail.com}{belk.stefan@gmail.com}}\\
Petz Thomas \\ {\small 0601280, 937, \href{mailto:e0601280@student.tuwien.ac.at}{e0601280@student.tuwien.ac.at}}\\
Causevic Alma \\ {\small 0847805, 534, \href{mailto:alma.causevic@hotmail.com}{alma.causevic@hotmail.com}}\\ 
Causevic Amra  \\ {\small 0649241, 534, \href{mailto:amra.causevic@hotmail.com}{amra.causevic@hotmail.com}}\\ 
Seebacher David \\ {\small 0327243, 534, \href{mailto:david.seebacher@student.tuwien.ac.at}{david.seebacher@student.tuwien.ac.at}}\\
\vspace{4cm}
}

\begin{document}

\begin{titlepage}
\maketitle

\end{titlepage}

% --------------------------------------------------------------

\thispagestyle{empty}
\tableofcontents
\pagebreak

\setcounter{page}{1}


% --------------------------------------------------------------

\note{
\textbf{Formal constraints}
\begin{itemize}
\item	  Font: Times New Roman oder Computer Modern (\LaTeX default)
\item    Fontsize: 11pt
\item     Single line spacing
\item     Margins: 2.5cm side and top/bottom
\item     \fbox{Language: ENGLISH}
\item    The proposal template should be filled incrementally. I.e., at the end there should be a full project proposal in a single PDF file.
\end{itemize}
\textbf{Available templates}
\begin{itemize}
\item     Proposal (mswp-proposal.tex)
\item     Costs (costs.xls, costs.ods)
\end{itemize}
\textbf{Supplemental material}
\begin{itemize}
\item     FWF salary scheme (\href{http://www.fwf.ac.at/de/projects/personalkostensaetze.html}{http://www.fwf.ac.at/de/projects/personalkostensaetze.html})
\item     Travel cost regulation (\href{http://www.fwf.ac.at/de/faq/reisegebuehrenvorschrift.html}{http://www.fwf.ac.at/de/faq/reisegebuehrenvorschrift.html})
\item     Ethical issues form (ethical-issues.rtf)
\end{itemize}
}
\pagebreak

% --------------------------------------------------------------
\section{Synopsis}
\label{sect:synopsis}
\subsection{Project Idea}
WaveMeIn is a research project to create a new type of secure login mechanism. It consists of a small device worn by the user at the ear which authenticates the user based on brain waves.

\subsection{Why do we need it?}
At the time of this proposal the most used ways for authentication are manually typed passwords or biometric authentication methods. However all of the previous methods have some security problems or are simply not user-friendly. Typed passwords are easy to spy out simply by looking at the keyboard of the user or the traces of the fingers on touch displays. In the case of biometric authentication, there are for example face recognition, iris or fingerprint scans. Face recognition software can easily be tricked by face masks or photographs and moreover depends on good light conditions, the quality of the images of the web camera and other factors. Fingerprint and iris scans are the most secure options of the authentication methods mentioned before. However they also have many disadvantages. Iris scans are not practical since the hardware required cannot easily be integrated into small devices and it is not user-friendly to require the user to place his eye very close to the scanner every time he/she wants to unlock a device. Fingerprint sensors are known to fail to recognize the fingerprint correctly quite often and it is also a not very user-friendly authentication method for handicapped people that may not reach the sensor or may not have any fingers at all. 

\subsection{How does it work?}
Brain waves are a secure and user-friendly alternative authentication method. The idea is to create a small device, called Wavy, that can be worn at the ear of the user in the same style as bluetooth headsets are already worn for communication today. The Wavy measures the brain waves near the ear in case a login is required by a client device that is connected via bluetooth. It listens for a brain wave pattern that was previously trained by the user as a password. If the correct pattern was detected by the Wavy it transmits a OK signal back to the client device. 

\subsection{Why should somebody care?}
Nowadays people are forced to type their passwords in public places which is a security risk and also not a very efficient way for authentication. Especially when typing in password on small devices such as mobile phones this authentication method is also very error prone due to the small keyboard interfaces. On the one side people are lazy and do not want to remember and enter long and complicated passwords, but on the other side they are also concerned about the security of their data and their privacy. So the users are in need of a more secure and easier way of authentication.

\subsection{Who are the beneficiaries of the results?}
Basically everybody can benefit from the WaveMeIn project since it is usable in the daily life. Especially for handicapped people it is a new and more easy to use option to log into their devices. Also it grants a higher level of security than existing authentication methods so it is also well suited for environments where higher security is needed, such as access authentication is modern research labs and government or military facilities.

For our product to succeed, we need to invest into research in the area of brain wave detection and analysis. This investment can improve our understanding of this topic. After a commercial success, we have to enhance our product. This means we have to invest further into brain wave research. On the other side, we can make our world more secure. It makes hacking of accounts and password fraud more complicated.

\subsection{Problem classification}
The task of detecting brain waves it tightly connected to the research areas of Neuroscience, Pattern Recognition and Machine Learning. In the field of Neuroscience it touches the areas of not invasive brain computer interfaces and neural oscillation. Since detecting and reliably identifying brain waves at the location near the ears is still technically immature the project can be seen as basic research in this area. The following research questions have to be answered before a prototype can be developed.
\begin{itemize}
	\item Detecting brain waves at the ears
	\item Recognize brain wave patterns
	\item Distinguish correct patterns from random signals
	\item Distinguish brain waves from different users
\end{itemize}
On the other hand if we take the Wavy into account, which should be the resulting product, this project is also an applied research project. It further touches the fields of computer security and privacy.



% --------------------------------------------------------------
\section{Introduction and problem description}
\label{sect:intro}
WaveMeIn is a research project to show the potential of brain waves a new method of electronic authentication via a small wearable device. Its aim is to investigate the usage of rain waves to replace passwords or other authentication methods. Therefore the properties of brain waves regarding uniqueness and reliability have to be explored. The project shall demonstrate via a small prototype that the recognition is possible without large sensors on top of the users head. A requirement for such a method of authentication clearly exists as demonstrated by the following use cases:

\subsection{Current Situation}


\subsection{Use Case 1}{
Assume a user needs to log on to a device (e.g. a notebook) that contains sensitive information in public. Typing in the password is not an option as it can easily be monitored by another person. Fingerprints are also not a good alternative as they can easily be taken from any surface the user touched and be copied onto synthetic materials to deceive the fingerprint reader. Brain waves are (as of current knowledge) unique for each person even if two people are having exactly the same thought. If the intruder does not know the precise brain wave pattern of the user's pass phrase/thought it is impossible to duplicate.}

\subsection{Use Case 2}{
Assume an average user wants to unlock his/her smart phone in a crowded area such as the subway. Nowadays this is done by entering a pin or drawing a pattern on the screen. A person with the intention to steal a users phone just needs to observe its victim while entering the pass code or pattern. Afterwards it is easy to unlock the phone and steal the victims personal data or cause large costs while using it for phone calls and mobile data. Locking the phone via a brain wave authentication mechanism may not prevent the theft but the costs arising from the phone being used afterwards.}

\subsection{Current Authentication Methods}
In theory brain waves will be rank among the most secure authentication methods, probably being the most secure one if the research proves successful. The particular brain wave of a user required to unlock a device can not be easily be obtained other than strapping the user to a chair and forcing him/her to think his/her pass thought. Other authentication methods are password, drawing patterns, fingerprint, iris scan, voice recognition. Passwords and pattern drawing are the least secure ones as the user can be observed while typing or drawing without much effort. Fingerprint, iris scan, voice recognition may require more technical or social effort to obtain, but in the end all of them are features of a person that are always visible for the outside world and therefore copyable with more or less effort.

\subsection{Unresolved Problems and Opportunities}
The unknown factor of this research project is that no research has been done on measuring brain waves at other locations (e.g. the ears) of the body except directly at the users head. Additionally it is unknown if the brain waves of are person are distinctive enough to distinguish a pass thought of a user from other thoughts and if the brain waves of different users while thinking the same thought are distinctive enough.

At the time of writing this proposal there exists no device that is capable of the features mentioned above as well as being small enough to be worn as an accessory. Therefore this is an important area of research with practical future applications.

\subsection{Terms}
\begin{itemize}
	\item 
	\item Brain waves:
	\item
\end{itemize}

% --------------------------------------------------------------
\section{Project goals and deliverables}
\label{sect:goals}
The following sections will provide an overview over the research questions and hardware questions associated with the project.
\subsection{Research questions}
\begin{itemize}
	\item How can be brain waves be detected by the a small device at a single location?
	\item How reliable is the detection of individual brain waves of the same person?
	\item How reliable is unique identification of the brain waves of different persons?
	\item Is it possible to detect brain waves at other body locations than the head?
\end{itemize}

\subsection{Hardware Design}
\begin{itemize}
	\item How can the required hardware be minimized to be small and practical (Wavy Device)?
	\item Are the existing sensors for measuring rain waves good enough for the projects requirements?
\end{itemize}

\subsection{Expected Results}
\begin{itemize}
	\item Successful research on the identification of brain waves.
	\item Algorithms to reliably identify brain wave patterns.
	\item Creation of a small prototype device capable of reading brain waves.
\end{itemize}

At the end of the project it should be clear if:
\begin{itemize}
	\item The detection of brain waves is possible at different locations of the body.
	\item The same thought produces a repeatable and reliable brainwave pattern. (Reliability)
	\item Different people have different patterns when thinking the same thought. (Uniqueness)
	\item Brain waves can be used as authentication method.
	\item The necessary can e integrated into a small device.
\end{itemize}

\subsection{Non-Goals}
\begin{itemize}
	\item No mind reading device
	\item No client software (just the brain wave research, hardware and interface)
	\item No design or usability study (just a prototype that works and is small enough)
	\item No end-user/consumer product (just a prototype)
\end{itemize}

% --------------------------------------------------------------
\section{Scientific relevance and innovative aspects}
\label{sect:relevance}

%\note{
%\begin{itemize}
%\item {\em Length: 1-2 pages}
%\item Why is the project scientifically interesting?
%\item Did others point out that this is an open question?
%\item What are the innovative aspects that make it interesting?
%\item How could the project break new ground scientifically?
%\item To what extent are the objectives ambitious and beyond the state of the art (e.g. novel concepts and approaches or development across disciplines)?
%\end{itemize}
%}
% % % % % % % % % % % % % % % % % % % % % % % % % % % % % % % % % % % % % %
WaveMeIn can be seen as an important step in the development of brain-computer interfaces used on a daily basis. Given the huge possibility, many private corporations as well as research institutes initiated promising projects. To get a quick overview see Section \ref{sect:star}.

\subsection{Simple step into real life}
The use case of brain wave controlled interaction in WaveMeIn is still kept simple, as it is used only to unlock a device and no further commands have to be recognized. On success, following projects can base more sophisticated ways of interaction on results of WaveMeIns research. Even if this project covers a lot of ground work as well, the focus is to create a product usable on a daily basis. Therefore it goes a little further then most other project in this field, as they concentrate on mostly one specific aspect.

\subsection{Brain wave recognition}
In the field of neuroscience the main question will probably be, what kind of brain waves produce recognizable patterns of the same imagination. Another question is, under what circumstances do brain wave scans look similar. Does the pattern change if the context of the person changes, like a noisy environment, strong emotions or the effect of drugs?
The link to pattern matching in computer science would be, how to match the original password-pattern, recorded in a probably neutral state and the input-pattern within a shifted context. This leads to the question, if a brain wave pattern can be normalized without knowledge of the specific context.

\subsection{BCI improvement}
In conjunction with electrical engineering the BCI itself should be revised. The goal is to shrink the scanner to a minimum so it does not disturb while wearing it for many hours in public. There not only size matters but the position of the scanner should be as flexible as possible. That said, a scanner with the proportions of a Bluetooth headset seems appropriate but not feasible at the moment. One task is to raise the level of detail of the scanners and in the same time to suppress undesired noise. It is still unclear what areas of the head are viable to work with an even improved non invasive BCI.

\subsection{Device security}
Since WaveMeIn is not only meant to simplify the unlock mechanism, but to raise the security of the procedure as well, this will be an important task in the area of computer science. The password itself is an interpretation of a specific thought and therefore never conventionally visible as maybe a fingerprint or a typed password. But still, it will be scanned, processed and the interpretation itself or at least a answer will be sent to the device that is waiting to get unlocked. The most vulnerable moment is during transport of the data. The security requirement should be comparable to wireless networks or Bluetooth connections and is therefore a well researched area already.

%* security: since the password in this case, the bw pattern is never shown in public, the transport from the sensor to the computer is the primary attack vector (!?) -> tight protocoll and gadget encryption is a must

%The task of detecting brain waves it tightly connected to the research areas of Neuroscience, Pattern Recognition and Machine Learning. In the field of Neuroscience it touches the areas of not invasive brain computer interfaces and neural oscillation. Since detecting and reliably identifying brain waves at the location near the ears is still technically immature the project can be seen as basic research in this area. The following research questions have to be answered before a prototype can be developed.
% % % % % % % % % % % % % % % % % % % % % % % % % % % % % % % % % % % % % % % % % % %

%http://www.technologyreview.com/news/513861/samsung-demos-a-tablet-controlled-by-your-brain/
%\em{http://www.essp.utdallas.edu/uploads/Main/Publications/WH2013\_Demo\_Paper.pdf}
% --------------------------------------------------------------

\section{State of the art / current knowledge}
\label{sect:star}

\note{
\begin{itemize}
\item {\em Length: 2-5 pages}
\item What results and approaches have already been presented in this or related areas?
\item Relation to the international scientific work in the field (international status of the research)
\item Description and critical discussion of related scientific work
\end{itemize}
}

% --------------------------------------------------------------
\section{Method}
\label{sect:method}
\subsection{How should the expected results be achieved?}
In the first place some basic research about brain waves should be done in order to achieve the expected results and build a device for brain wave based authentication. The goal of the projects research part is to clarify the scientific requirements needed for the creation of the prototype, for example to confirm the uniqueness of brain waves or define how specific brain waves of a person can be identified. This is important since we want to assure that the user can be identified as the same one later again. It should be guaranteed that thoughts provide a reliable identification method, therefore the research should give answers how reliably brain waves of a person can be identified.

The research part consists of two phases – a theoretical and an empirical research part. The purpose of the theoretical part is to analyse existing studies and use the results for the project. Already existing studies confirmed the possibility of authentication via brain waves, but so far no study managed to measure brain waves from different head parts than the forehead. The studies showed theoretical approaches to measure brain waves using EEG electrodes placed at the users forehead. The goal of the theoretical research part is to use this approaches and sensor minimising methods in order to define requirements for building a small device, called Wavy. Therefore, the goal to measure brain waves from different head or even body parts is a very important part of the research. The Wavy device should be as small as possible and designed in a way which allows the user to wear it in everyday life. The ears will be the primary focus of this research for measuring brain waves at positions less visible then the forehead. This will enable the Wavy device to be worn like an accessory on or behind the ear. 

In the empirical part there will be some practical tests with human beings, where their brain waves will be measured with an EEG in different situations. The goal of this research part is to confirm the uniqueness of two peoples brain waves, even if they think about the same thing. The approach from UC Berkeley, which John Chuang presented at the 17th International Conference on Financial Cryptography and Data Security, is a good one to use for the project. In this approach the participants were asked to perform seven mental tasks. These were divided into two categories. For the first group of tasks - which was the same for all participants - the subjects were asked to do simple things, for example to focus on their own breathing, imagine moving a finger, or to listen for an audio tone and respond to that by focusing on a dot on a piece of paper. For the second group of tasks - which participants selected and performed individually without letting others know what they were doing - the subjects were asked to select from imagining performing a repetitive motion from their favourite sport, such as swinging a golf club; singing a song of their choice; watch a series of on-screen objects and silently count those that matched a color of their choice; or think of something for 10 seconds. \cite{Adhikari13} With such tests the Berkeley researchers managed to ensure that brain waves provide enough information to authenticate the user's identity. This kind of tests will be also done with users in our empirical research part once the prototype will be constructed.

\subsection{What methods will be applied?}
The following methods will be used:

\subsubsection{Theoretical study}
The goal of the theoretical study is to gather information from existing studies on this field. Several different studies already investigated the possibility of using brain waves as access method. Their findings will be used as a start point for our research. Researchers in the field of neuroscience and computer science will be needed for this task which will be part of the first work packages at the beginning of the project.

\subsubsection{Experiments}
This part consists of performing practical tests with participants. During these tests the users will wear an EEG device and will be asked to think of some simple things as well as thinking of some personal memories or for example their favourite song. The experiments will define whether brain waves of every single person are unique and how reliable they can be measured from different body parts. The participants will have to be selected as heterogeneous group so that the experiments can be performed on people with different characteristics such as male/female, adult/children, etc. Again the neuroscience and computer science are the required fields of expertise for this part.

\subsubsection{Prototype}
In order to get to this project phase the first phase which includes the brain wave connected research part should be completed. The goal of this phase is to use the results from the research part and create the Wavy prototype and write a demo software. The Wavy should be as small as possible and communicates with the user's device via Bluetooth. It listens for a brain wave pattern that was previously trained by the user as a password. If the correct pattern was detected by the Wavy it transmits a OK signal back to the client device. A demonstration software shall be written to show how the Wavy device can be used. This work will require an hardware and a software engineer to work together with the researchers.

\subsubsection{Scientific study}
At the end of the project the security and usability of authenticating via the Wavy device will be compared to existing authentication methods. This study should generate a comprehensive report which can be published to show the advantages of this new authentication method and raise the public's awareness for it.


% --------------------------------------------------------------
\section{Detailed description of the workpackages}
\label{sect:workplan}

\note{
\begin{itemize}
\item {\em Length: 2-4 pages}
\item Structuring the project into self-contained parts.
\item Additional verbal descriptions.
\item Work packages
    \begin{itemize}
    \item title
    \item goal(s)
    \item description
    \item expected results
    \item responsible person(s)
    \item dependencies
    \end{itemize}
\end{itemize}
}
% --------------------------------------------------------------
\section{Time plan (Gantt chart)}
\label{sect:timeplan}
As specified in section 7, the project consists of three basic work packages (each of them having several sub-packages):
\begin{itemize}
\item WP 1 - Basic research
\item WP 2 - Hardware and software development
\item WP 3 - Prototype development
\end{itemize}

\begin{figure}[H]
	\centering
    \includegraphics[width=1\textwidth]{gantt_chart}
    \caption{GANTT Chart}
\end{figure}

Each of these packages consists of another sub-packages. Since the research part of the project delivers the basis for the whole project and therefore is the most important part of the project, it has a planned duration of 1,5 years. The planned duration of the hardware and software development package is 1 year and the duration of the prototype development package 1 year. Also there is a scheduled time buffer of one month in each package, which can be used in potential problematic situations and in cases of time delays appearances. 

\subsection{Milestones}
The planned milestones of the projects are:
\begin{itemize}
\item 05.07.2015 - Confirmed uniqueness of a person’s brain wave pattern
\item 05.07.2016 - Research part completed 
\item 05.01.2017 - Hardware and Software for Microcontroller finished
\item 05.06.2017 - Hardware Part Completed 
\item 05.02.2018 - Prototype Completed 
\item 05.07.2018 - Project Completed 
\end{itemize}

\subsection{Timeplan}
The estimation of the schedule based on work packages:

\subsubsection{WP1 - Basic research}
\begin{itemize}
\item start date: 05.01.2015
\item end date:  05.07.2016
\item milestone: Research part completed
\end{itemize}


\subsubsection{WP 1.1 - Confirm uniqueness of a person’s brain wave pattern}
\begin{itemize}
\item start date: 05.01.2015
\item end date: 05.07.2015
\item milestone: Confirmed uniqueness of a person’s brain wave pattern
\end{itemize}
\subsubsection{WP 1.2 - Reliable thought identification}
\begin{itemize}
\item start date: 05.07.2015
\item end date:  05.01.2016
\end{itemize}
\subsubsection{WP 1.3 - Measuring methods of brain waves from different parts of the head}
\begin{itemize}
\item start date: 05.01.2016
\item end date: 05.07.2016
\end{itemize}
\subsubsection{WP 1.4 - Survey of state of the art for sensors and devices for brain wave measurement}
\begin{itemize}
\item start date: 05.01.2015
\item end date: 05.01.2016
\end{itemize}
\subsubsection{WP 2 - Hardware and software development}
\begin{itemize}
\item start date: 05.07.2016
\item end date:  05.07.2017
\end{itemize}

The Milestones for the work packages 2.2, 2.3 and 2.4 share the same milestone - Hardware and Software for Microcontroller finished and after this three work packages there is also planned a buffer of one month for eventual time delays.

\subsubsection{WP 2.1 - Reviewing of existing algorithms for detecting patterns in measured brain}
\begin{itemize}
\item start date: 05.07.2016
\item end date:  05.10.2016
\end{itemize}
\subsubsection{WP 2.2 - Develop new algorithms for detecting patterns in measured brain waves}
\begin{itemize}
\item start date: 05.10.2016
\item end date:  05.01.2017
\item milestone: Hardware and Software for Microcontroller finished 
\end{itemize}
\subsubsection{WP 2.3 - Define the communication protocol and interface of the Wavy}
\begin{itemize}
\item start date: 05.07.2016
\item end date:  05.07.2017
\item milestone: Hardware and Software for Microcontroller finished 
\end{itemize}
\subsubsection{WP 2.4 - Develop small sensors for brain wave measurement}
\begin{itemize}
\item start date: 05.07.2016
\item end date: 05.01.2017
\item milestone: Hardware and Software for Microcontroller finished 
\end{itemize}
\subsubsection{WP 2.5 - Develop a microcontroller for the Wavy}
\begin{itemize}
\item start date: 05.01.2017
\item end date: 05.07.2017
\item milestone: Hardware Part Completed
\end{itemize}
\subsubsection{WP 3 - Prototype development}
\begin{itemize}
\item start date: 05.07.2017
\item end date: 05.07.2018
\item milestone: Project Completed
\end{itemize}
\subsubsection{WP 3.1  - Create the Wavy prototype}
\begin{itemize}
\item start date: 05.07.2017
\item end date: 05.03.2018
\item milestone: Prototype Completed
\end{itemize}
\subsubsection{WP 3.2  - Write a software prototype}
\begin{itemize}
\item start date: 05.07.2017
\item end date: 05.03.2018
\item milestone: Prototype Completed
\end{itemize}
\subsubsection{WP 3.3  - Comparing with existing authentication methods}
\begin{itemize}
\item start date: 05.03.2018
\item end date:  05.07.2018
\end{itemize}

\subsection{Critical Areas}
The critical areas of this project are in the research part, because if the research parts doesn’t deliver the following results, the project will not fail but it won’t lead to the planned project goals. The Work package 1.1 (Confirm uniqueness of a person’s brain wave pattern) and 1.3 (Identify the methods to measure brain waves from different head parts) are the most critical parts of the project.






% --------------------------------------------------------------
\section{Human resources / team}
\label{sect:team}

\note{
\begin{itemize}
\item {\em Length: 1-2 pages}
\item Description of the team that is needed to carry out the project. (For the execution phase of the project, not the planning phase.)
\item How many people?
\item To what extent are individual members needed?
\item What knowledge, skills, and experiences are needed for each member?
\item Demonstrate that the members will be able to carry out the project successfully.
\item Work structure
	\begin{itemize}
	\item     Who will lead the project?
	\item     How do they work together?
	\item     Management and coordination
		\begin{itemize}
		\item 	        What communication structures will be established? (e.g., mailing list, blog, CMS, CVS, ...)
		\item 	        How often will meetings take place? (Who will participate?)
		\item 	        How will the work be documented?
		\item 	        How will information be stored and shared?
		\end{itemize}
	\end{itemize}
\item Cooperations
	\begin{itemize}
	\item     Will external cooperators be part of the project? (e.g., other research institutions or companies)
	\item     What is their role?
	 \item    Why are they needed?
	\end{itemize}
\end{itemize}
}
\subsection{Required Persons and Roles}
\begin{itemize}
\item Project Manager
\item Neuroscience researcher
\item Computer science researcher
\item Hardware engineer
\item Software engineer
\end{itemize}



% --------------------------------------------------------------
\section{Costs}
\label{sect:costs}

\note{
\begin{itemize}
\item {\em Length: 2-3 pages}
\item Rough estimation of cost in form of calculation (table(s)) + descriptive text.
\item Justification for the personnel and non-personnel costs (equipment, material, travel and other costs)
\item An Excel template is provided as supplementary material to support budgeting.
\item Personnel costs
	\begin{itemize}
	\item     Justification for the personnel to be assigned to the project (type of position(s), description of nature of work, length and extent of involvement in the project)
	\item     The application should include all persons who will be required for the proposed project (project lead, researchers, developers, advisory board, etc.). The available legal categories of employment are contracts of employment for full- or part-time employees (DV) and reimbursement for work on an hourly basis (GB). In addition, a part-time contract of employment (DV 50\%, ``studentische Mitarbeiter'') may be requested for people who have not yet completed a Master or Diploma program (Diplom) in the relevant subject.
	 \item    The justification of the requested personnel should contain:
		\begin{itemize}
		\item 	        description of type of work;
		\item 		        extent of involvement (part-time contracts are permitted).
		\end{itemize}
	\item Exact numbers of employment categories can be found on the FWF Website (\href{http://www.fwf.ac.at/de/projects/personalkostensaetze.html}{http://www.fwf.ac.at/de/projects/personalkostensaetze.html})
	\end{itemize}
\item Equipment costs
	\begin{itemize}
	\item     Indicate reasons for equipment costs. The ``scientific equipment'' category includes instruments, system components, costs for the use of software required by the project and other durable goods provided the cost per item (including VAT) exceeds EUR 1,500.00.
	\end{itemize}
\item Material costs
	\begin{itemize}
	\item     This category encompasses consumables and smaller pieces of equipment where the cost per item is below EUR 1,500.00 including VAT. The calculation of requested material costs should be justified with reference to the schedule, work plan and experimental plan. Experience with previous projects should be taken into account.
	\end{itemize}
\item Travel costs
	\begin{itemize}
	\item     Funding may be requested for the costs of project-specific travel and accommodation, field work, expeditions, etc. Applicants are to provide a detailed travel (cost) plan broken down by project participant. For brief stays, the calculation of the travel and accommodation costs should be based on the federal regulations governing travel costs (RGV). The RGV rates governing Austria and abroad may be found in the FAQs on the FWF Website (\href{http://www.fwf.ac.at/de/faq/reisegebuehrenvorschrift.html}{http://www.fwf.ac.at/de/faq/reisegebuehrenvorschrift.html}). For longer stays an appropriate and comprehensible cost plan should be prepared.
	\end{itemize}
\item Other costs
	\begin{itemize}
	\item     Independent contracts for work and services (costs for work of clearly defined scope and content assigned to individuals, provided that this is scientifically justifiable and economical)
    	\item     Costs that cannot be included under personnel, equipment, material or travel costs, such as:
		\begin{itemize}
		\item         reimbursement of costs towards or for the use of research facilities, e.g. of large-scale research facilities (project-specific 'equipment time'). Applicants should obtain and submit multiple offers;
		\item         costs for project-specific work carried out outside the applicant's research institution (e.g. for analysis work performed elsewhere, for interviews, for sample collection, for preparation of thin slices etc.). Applicants should obtain and submit multiple offers;
		\item         honoraria for test persons;
		\end{itemize}
	\end{itemize}
\end{itemize}
}


% --------------------------------------------------------------
\section{Expected implications and risks}
\label{sect:implication-risk}

\note{
\begin{itemize}
\item {\em Length: 1-2 pages}
\item Importance of the expected results for the discipline
	\begin{itemize}
	\item     To what extent does the proposed research address important challenges?
	\end{itemize}
\item Importance of the expected results for other areas
\item What are possible risks of the project and how can they be alleviated?
	\begin{itemize}
	\item     What factors could lead to a failure of the project?
	\item     Which factors or persons could support the project and increase the chance for success?
	\item     What if important team members leave the project?
	\end{itemize}
\end{itemize}
}


% --------------------------------------------------------------
\section{Ethical considerations \& security issues}
\label{sect:ethics-security}

\note{
\begin{itemize}
\item {\em Length: 1-2 pages}
\item Provide a brief explanation of the ethical issue involved and how it will be dealt with appropriately.
\item Are there any security-sensitive issues that apply to your proposal?
\end{itemize}
}


% --------------------------------------------------------------
% APPENDIX
\begin{appendix}

\pagebreak

% --------------------------------------------------------------
% References
\phantomsection
\addcontentsline{toc}{section}{References}

\bibliographystyle{apalike}
\bibliography{mswp2013-proposal}

\pagebreak

% --------------------------------------------------------------
% Abbreviations
\section*{Abbreviations}
 \addcontentsline{toc}{section}{Abbreviations}
 
 \begin{description}
  \item[MSWP] Management von Software Projekten
  \item[WP] Work Package
 \end{description}

\end{appendix}


\end{document}
