\section{Ethical Considerations \& Security Issues}
\label{sect:ethics-security}
\subsection{Ethical Considerations}

\subsubsection*{Electromagnetic Waves}
At the moment there is an ongoing dispute in the public over weather or not long-term exposure to electromagnetic fields can cause cancer or changes in the brain chemistry. As far as we know today weak electromagnetic fields will have no effects on the human brain, but strong fields definitively do. The Wavy device will definitely create such a weak magnetic field (like all electric devices) and due to its location at the users ear and the intention of wearing it permanently it has the potential of harming the user if electromagnetic waves really have negative effects on the human brain. 

\subsubsection*{Collecting User Data}
The Wavy device could be copied by companies trying to make a profit by constantly monitoring the emotional state of the user. The device would still function as an access device, but it could be misused for marketing purposes for example. A possible scenario could be: A company installs a wireless module into a TV set that sends a signal when an advertisement is shown. This signal activates an application in the user’s phone which in turn starts collecting emotional data via such a “bad” Wavy device from the user who is watching the advertisement. The company can then sell the gathered knowledge about the user’s response to the advertisement and his/her preferences to others such as the advertising company.

\subsection{Security Issues}
\subsubsection*{Mind Reading}
At the current state of the technology mind reading is still science fiction, but basically a device such as the Wavy could be used to read all thoughts of a user. The only real issue that is preventing this scenario at the moment is that from a given set of brain waves there is currently no way of translating this user’s thought into a visual or textual representation for someone else to read. However, this may be possible at some point in the future. A compromised Wavy device poses a serious security risk for the user.

\subsubsection*{Bluetooth}
The biggest issue in security is probably the Bluetooth connection itself. We have to rely on the Bluetooth standard to be secure, as we cannot change it and there are basically no other alternatives to Bluetooth at the moment.

