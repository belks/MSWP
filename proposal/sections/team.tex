\section{Human resources / team}
\label{sect:team}
\subsection{Required Persons and Roles}
\begin{itemize}
\item Project Manager
\item Neuroscience Researcher
\item Computer Science Researcher
\item Hardware Engineer
\item Software Engineer
\end{itemize}

\subsection{Project Manager}
The project managers have the responsibility of the planning, execution and closing of the project. A project manager is the person responsible for accomplishing the stated project objectives. Key project management responsibilities include creating clear and attainable project objectives, building the project requirements, and managing the constraints of the project management triangle, which are cost, time, scope and quality. He is the bridging gap between the production team and client. So he/she must have a fair knowledge of the industry they are in so that they are capable of understanding and discussing the problems with either party.

However, there are some responsibilities that are common to all project Managers, noting:
\begin{itemize}
\item Developing the project plan
\item Managing the project stakeholders
\item Managing Communication
\item Managing the project team
\item Managing the project risk
\item Managing the project schedule
\item Managing the project budget
\item Managing the project conflicts
\item Managing the project delivery
\end{itemize}
So the project manager needs the following skills:
\begin{itemize}
\item Knowledge of project management
\item General management knowledge
\item Product-specific knowledge
\item Stamina and Endurance
\item A holistic and sustainable way of thinking
\item Interpersonal and communication skills
\end{itemize}

\subsection{Neuroscience Researcher}
The main tasks of the neuroscience researcher are to prove what kind of brain waves produce recognizable patterns of the same imagination, under what circumstances brain wave scans look similar and to research if the pattern changes if the context of the person changes, like a noisy environment, strong emotions or the effect of drugs.
The neuroscience researcher works in the first time period of the project, from the 05.01.2015, where the basic research about brain waves should be done in order to achieve the expected results. There are two main scientific questions that need to be answered by the neuroscience researcher.
\begin{itemize}
\item{
First of all he should confirm or disprove the uniqueness of brain waves or define how specific brain waves of a person can be identified.
}
\item{
The second task of the researcher is to confirm the uniqueness of a persons brain waves compared with the brain waves of others. Another important factor is to confirm that the brain waves of a specific thought of a person are distinguishable from other thoughts of the same person.
}
\item{
The prototype Wavy should make it possible to indirectly watch the brains function. The activity of the neurons generates an electric field, which can be measured from  outside the skull.
}
\end{itemize}

\subsection{Hardware Engineer}                         	
The hardware engineer begins his work at the beginning of the first phase of the project and starts simultaneously with the neuroscience researcher. There already are lots of existing sensors for detecting brain waves. The main task of the hardware engineer is to look if a suitable device already exists and can be made smaller or if he has to develop a new device. He is responsible for the development of the prototype in order to show that authentication over brain waves works. Also, he needs technical knowledge about signal processing.

The goal of his work is to develop a small as possible prototype device, which makes it possible for the user to wear it in the everyday life. For that purpose, the project is in need of small as possible sensors, which will be developed by the hardware engineer during his work.

He also needs to develop a micro-controller, which manages the detection and identification of brain waves. It is working with the developed algorithm and sending it over a Bluetooth interface to a chosen device. The communication with the controller is handled with the developed software interface. Therefore, the software engineer and the hardware engineer are working closely together in the same time period.

\subsection{Computer Science Researcher}
The computer researcher identifies a persons thoughts and reviews existing algorithms for detecting patterns in measured brain waves. Since it is possible that the same thought of a person can be shown different in the brain wave, we need some patterns to confirm the measured brain waves implemented in a working algorithm with good performance. If there are not found existing algorithms the development of new ones is the goal of the computer science researcher. His tasks is also to compare the results with existing authentication methods.

The computer science researcher should have knowledge of machine learning, pattern detection, human computer interface design and signal processing. He should have worked in research projects already, where he has experienced some work with brain wave research systems.

\subsection{Software Engineer}
He is a person concerned with facets of the software development process. A software  developer may take part in design, computer programming, or software project management. They may contribute to the overview of the project on the application level rather than  component-level or individual programming tasks.  His task in the project is to work on the communication protocols and the interface of the Wavy device.

This interface should make it possible for the operation system or other developers to handle their authentication over the Wavy. He is also responsible for the prototype software, which is used by the WaveMeIn.
The software engineers working area may include:
\begin{itemize}
\item Software design
\item Implementation
\item Requirement analysis
\item {Testing, including defining/supporting acceptance testing and gathering feedback from pre-release testers}
\item {Participation in software release and post-release activities, including support for product launch evangelism}
\item {Maintenance}
\end{itemize}

\subsection{Quality Assurance Manager}
The quality assurance manager has tasks in the following areas:
\begin{itemize}
\item Controlling
\item Testing
\item Reviews
\item Developing and evaluating statistics
\end{itemize}

Controlling - the quality assurance manager controls and monitors the progress of the process, he checks if plans are realistic and if a functioning project controlling exists.

Testing - the quality assurance manager checks if the test plan is conform with the quality assurance requirements, he checks if the planed tests have been executed and he monitors the test metrics.

Reviews - the quality assurance manager has to have review-knowledge. In reviews he often plays the role of a facilitator, because he checks if the found errors have been corrected after the review process. He also checks if the planed reviews have been executed and he is responsible for the quality of the executed review.

\subsection{Work Structure}
The leader of the project is the project manager, he is responsible for the achievement of project goals, resource goals and timing goals. He is also responsible for the management and coordination of the work. The team communicates via mailing lists, scrum board and GIT. Weekly there will be SCRUM meetings and daily mini scrum meetings with duration of max. 15 minutes. Once a month meetings with all project members will be held and reviews and the work progress will be discussed.The work and the information will be stored and shared with GIT. External cooperators will not be part of the project.
