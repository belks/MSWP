\section{Scientific relevance and innovative aspects}
\label{sect:relevance}

\note{
\begin{itemize}
\item {\em Length: 1-2 pages}
\item Why is the project scientifically interesting?
\item Did others point out that this is an open question?
\item What are the innovative aspects that make it interesting?
\item How could the project break new ground scientifically?
\item To what extent are the objectives ambitious and beyond the state of the art (e.g. novel concepts and approaches or development across disciplines)?
\end{itemize}
}
% % % % % % % % % % % % % % % % % % % % % % % % % % % % % % % % % % % % % %
WaveMeIn is an important step in the development of brain-computer interfaces used on a daily basis. Given the huge possibility, many private corporations as well as research institutes initiated promising projects ***hier ein paar links aus state of the art...***

The use case of brain wave controlled interaction in WaveMeIn is still kept simple as it is used only to unlock and no further commands have to be recognized. On success, following projects can base more sophisticated ways of interaction on results of WaveMeIns research.


* project touches several fields like computer science, neuroscience or even industrial design.
* many projects in this area, often in cooperation with industry <- results could change interaction completely
 
 general:
* foundation to more sophisticated applications of brainwave based interaction concepts
* a (pretty) simple way to show brainwave based interaction and pushes the public and scientific  interest
 
field of neuroscience/computerscience:
* do patterns differ depending on kontext (noisy environment, sleepy person or under influence of drugs/medication/zustandsverändernde substanzen)
* are the differences calculable, means, can a normalized pattern be calculated from an unknown kontext?

field of neuroscience and hw -design!!?!??!:
* optimize brain-computer-interface: smaller in size - possible to find all required bw patterns on one spot on the head?, position-independent - is it possible to find pattern in an area where its feasible to use a everyday weareble as bci? (eg not forehead but ear)


computer science ->pattern recognition (in welchen wissenschaftlichen bereichen relevant/beheimatet?) 
* pattern recognition already good, probabbly different approach needed if sensors do not record the same amount or as detailed as with bigger bci..? ->sinnvoll?!
* security: since the password in this case, the bw pattern is never shown in public, the transport from the sensor to the computer is the primary attack vector (!?) -> tight protocoll and gadget encryption is a must

The task of detecting brain waves it tightly connected to the research areas of Neuroscience, Pattern Recognition and Machine Learning. In the field of Neuroscience it touches the areas of not invasive brain computer interfaces and neural oscillation. Since detecting and reliably identifying brain waves at the location near the ears is still technically immature the project can be seen as basic research in this area. The following research questions have to be answered before a prototype can be developed.
% % % % % % % % % % % % % % % % % % % % % % % % % % % % % % % % % % % % % % % % % % %

%http://www.technologyreview.com/news/513861/samsung-demos-a-tablet-controlled-by-your-brain/
\em{http://www.essp.utdallas.edu/uploads/Main/Publications/WH2013\_Demo\_Paper.pdf}