\section{Scientific relevance and innovative aspects}
\label{sect:relevance}

%\note{
%\begin{itemize}
%\item {\em Length: 1-2 pages}
%\item Why is the project scientifically interesting?
%\item Did others point out that this is an open question?
%\item What are the innovative aspects that make it interesting?
%\item How could the project break new ground scientifically?
%\item To what extent are the objectives ambitious and beyond the state of the art (e.g. novel concepts and approaches or development across disciplines)?
%\end{itemize}
%}
% % % % % % % % % % % % % % % % % % % % % % % % % % % % % % % % % % % % % %
WaveMeIn can be seen as an important step in the development of brain-computer interfaces used on a daily basis. Given the huge possibility, many private corporations as well as research institutes initiated promising projects. To get a quick overview see Section \ref{sect:star}.

\subsection{Simple step into real life}
The use case of brain wave controlled interaction in WaveMeIn is still kept simple, as it is used only to unlock a device and no further commands have to be recognized. On success, following projects can base more sophisticated ways of interaction on results of WaveMeIns research. Even if this project covers a lot of ground work as well, the focus is to create a product usable on a daily basis. Therefore it goes a little further then most other project in this field, as they concentrate on mostly one specific aspect.

\subsection{Bain wave recognition}
In the field of neuroscience the main question will probably be, what kind of brain waves produce recognizable patterns of the same imagination. Another question is, under what circumstances do brain wave scans look similar. Does the pattern change if the context of the person changes, like a noisy environment, strong emotions or the effect of drugs?
The link to pattern matching in computer science would be, how to match the original password-pattern, recorded in a probably neutral state and the input-pattern within a shifted context. This leads to the question, if a brain wave pattern can be normalized without knowledge of the specific context.

\subsection{BCI improvement}
In conjunction with electrical engineering the BCI itself should be revised. The goal is to shrink the scanner to a minimum so it does not disturb while wearing it for many hours in public. There not only size matters but the position of the scanner should be as flexible as possible. That said, a scanner with the proportions of a Bluetooth headset seems appropriate but not feasible at the moment. One task is to raise the level of detail of the scanners and in the same time to suppress undesired noise. It is still unclear what areas of the head are viable to work with an even improved non invasive BCI.

\subsection{Device security.. todo}
Since WaveMeIn is not only meant to simplify the unlock mechanism, but to raise the security of the procedure as well, this will be an important task in the area of computer science. The password itself is a interpretation of a specific thought and therefore never conventionally visible as maybe a fingerprint or a typed password. But still, it will be scanned, processed and the interpretation itself or at least a answer will be sent to the device that is waiting to get unlocked. The most vulnerable moment is during transport of the data. The security requirement should be comparable to wireless networks or Bluetooth connections and is therefore a well researched area already.

%* security: since the password in this case, the bw pattern is never shown in public, the transport from the sensor to the computer is the primary attack vector (!?) -> tight protocoll and gadget encryption is a must

%The task of detecting brain waves it tightly connected to the research areas of Neuroscience, Pattern Recognition and Machine Learning. In the field of Neuroscience it touches the areas of not invasive brain computer interfaces and neural oscillation. Since detecting and reliably identifying brain waves at the location near the ears is still technically immature the project can be seen as basic research in this area. The following research questions have to be answered before a prototype can be developed.
% % % % % % % % % % % % % % % % % % % % % % % % % % % % % % % % % % % % % % % % % % %

%http://www.technologyreview.com/news/513861/samsung-demos-a-tablet-controlled-by-your-brain/
%\em{http://www.essp.utdallas.edu/uploads/Main/Publications/WH2013\_Demo\_Paper.pdf}