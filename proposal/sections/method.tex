\section{Method}
\label{sect:method}
\subsection{How should the expected results be achieved?}
In the first place some basic research about brain waves should be done in order to achieve the expected results and build a device for brain wave based authentication. The goal of the projects research part is to clarify the scientific requirements needed for the creation of the prototype, for example to confirm the uniqueness of brain waves or define how specific brain waves of a person can be identified. This is important since we want to assure that the user can be identified as the same one later again. It should be guaranteed that thoughts provide a reliable identification method, therefore the research should give answers how reliably brain waves of a person can be identified.

The research part consists of two phases – a theoretical and an empirical research part. The purpose of the theoretical part is to analyse existing studies and use the results for the project. Already existing studies confirmed the possibility of authentication via brain waves, but so far no study managed to measure brain waves from different head parts than the forehead. The studies showed theoretical approaches to measure brain waves using EEG electrodes placed at the users forehead. The goal of the theoretical research part is to use this approaches and sensor minimising methods in order to define requirements for building a small device, called Wavy. Therefore, the goal to measure brain waves from different head or even body parts is a very important part of the research. The Wavy device should be as small as possible and designed in a way which allows the user to wear it in everyday life. The ears will be the primary focus of this research for measuring brain waves at positions less visible then the forehead. This will enable the Wavy device to be worn like an accessory on or behind the ear. 

In the empirical part there will be some practical tests with human beings, where their brain waves will be measured with an EEG in different situations. The goal of this research part is to confirm the uniqueness of two peoples brain waves, even if they think about the same thing. The approach from UC Berkeley, which John Chuang presented at the 17th International Conference on Financial Cryptography and Data Security, is a good one to use for the project. In this approach the participants were asked to perform seven mental tasks. These were divided into two categories. For the first group of tasks - which was the same for all participants - the subjects were asked to do simple things, for example to focus on their own breathing, imagine moving a finger, or to listen for an audio tone and respond to that by focusing on a dot on a piece of paper. For the second group of tasks - which participants selected and performed individually without letting others know what they were doing - the subjects were asked to select from imagining performing a repetitive motion from their favourite sport, such as swinging a golf club; singing a song of their choice; watch a series of on-screen objects and silently count those that matched a color of their choice; or think of something for 10 seconds. \cite{Adhikari13} With such tests the Berkeley researchers managed to ensure that brain waves provide enough information to authenticate the user's identity. This kind of tests will be also done with users in our empirical research part once the prototype will be constructed.

\subsection{What methods will be applied?}
The following methods will be used:

\subsubsection{Theoretical study}
The goal of the theoretical study is to gather information from existing studies on this field. Several different studies already investigated the possibility of using brain waves as access method. Their findings will be used as a start point for our research. Researchers in the field of neuroscience and computer science will be needed for this task which will be part of the first work packages at the beginning of the project.

\subsubsection{Experiments}
This part consists of performing practical tests with participants. During these tests the users will wear an EEG device and will be asked to think of some simple things as well as thinking of some personal memories or for example their favourite song. The experiments will define whether brain waves of every single person are unique and how reliable they can be measured from different body parts. The participants will have to be selected as heterogeneous group so that the experiments can be performed on people with different characteristics such as male/female, adult/children, etc. Again the neuroscience and computer science are the required fields of expertise for this part.

\subsubsection{Prototype}
In order to get to this project phase the first phase which includes the brain wave connected research part should be completed. The goal of this phase is to use the results from the research part and create the Wavy prototype and write a demo software. The Wavy should be as small as possible and communicates with the user's device via Bluetooth. It listens for a brain wave pattern that was previously trained by the user as a password. If the correct pattern was detected by the Wavy it transmits a OK signal back to the client device. A demonstration software shall be written to show how the Wavy device can be used. This work will require an hardware and a software engineer to work together with the researchers.

\subsubsection{Scientific study}
At the end of the project the security and usability of authenticating via the Wavy device will be compared to existing authentication methods. This study should generate a comprehensive report which can be published to show the advantages of this new authentication method and raise the public's awareness for it.

