\documentclass[a4paper,11pt]{article}

\usepackage{graphicx}
\usepackage{natbib}
\usepackage[utf8]{inputenc}
\usepackage{tabularx}
\usepackage{hyperref}
\usepackage{color}
\usepackage[usenames,dvipsnames,svgnames,table]{xcolor}
% \usepackage{mathptmx} % Times New Roman

\setlength{\topmargin}{-0.4mm} % (1in=25.4mm)-0.4mm=25mm
\setlength{\textheight}{243.119mm} % 297mm-40mm-10mm-(11pt=3.881mm)=
\setlength{\oddsidemargin}{-0.4mm} % (1in=25.4mm)-0.4mm=25mm
\setlength{\textwidth}{160mm} % 210mm-50mm=160mm
\setlength{\headheight}{0mm}
\setlength{\headsep}{0mm}
\setlength{\footskip}{15mm}

\providecommand*{\note}[1]{\small \textcolor{RoyalBlue}{\begin{minipage}{\textwidth}{#1}\end{minipage}}}

% --------------------------------------------------------------

\providecommand*{\ShortTitle}{WaveMeIn}
\providecommand*{\FullTitle}{Projektprotokoll}
\providecommand*{\includeWiki}[1]{\wikiEnvironments \include{#1}\nowikiEnvironments}

% --------------------------------------------------------------

\title{\textbf{\sffamily\Huge \ShortTitle}\\ 
{\textbf{\sffamily\Large \FullTitle}}
\vspace{1cm}}

\author{
{\em 188.407: Management von Software Projekten} \vspace{1cm} \\
Group: 10\bigskip \\
Belk Stefan \\ {\small 0750926, 937, \href{mailto:belk.stefan@gmail.com}{belk.stefan@gmail.com}}\\
Petz Thomas \\ {\small 0601280, 937, \href{mailto:e0601280@student.tuwien.ac.at}{e0601280@student.tuwien.ac.at}}\\
Causevic Alma \\ {\small 0847805, 534, \href{mailto:alma.causevic@hotmail.com}{alma.causevic@hotmail.com}}\\ 
Causevic Amra  \\ {\small 0649241, 534, \href{mailto:amra.causevic@hotmail.com}{amra.causevic@hotmail.com}}\\ 
Seebacher David \\ {\small 0327243, 534, \href{mailto:david.seebacher@student.tuwien.ac.at}{david.seebacher@student.tuwien.ac.at}}\\
\vspace{4cm}
}

\begin{document}

\begin{titlepage}
\maketitle

\end{titlepage}

% --------------------------------------------------------------

\section{Submission 2 -Treffen Freihaus am 20.11.2014}

\begin{itemize}
\item
  \textbf{Anwesende:} Alle
\item
  \textbf{Abwesend:} Niemand
\item
  \textbf{Ort:} Freihaus, Arbeitsraum gelb
\item
  \textbf{Zeit:} 13:00 - 17:00
\item
  \textbf{Tagesordnung:} Gemeinsame Diskussion für die Kapitel 2 bis 5
\begin{itemize}
  \itemsep1pt\parskip0pt\parsep0pt
  \item Goals, Non Goals abgegrentzt
  \item Resaerch questions formuliert
  \item Arbeitsaufteilung
\end{itemize}
\end{itemize}  
  
\section{Weiterer Verlauf: Gegenseitige Fortschrittsberichte und Hilfestellung über Emails.}

\subsection{21.11.2014 - Stefan}
Hi,

Also wir sind wie erwartet nicht ganz fertig geworden, aber haben schon einiges geschafft.
Alma und Amra bitte schaut euch Kapitel 2 an und erarbeitet Kapitel 5.

Kapitel 2)
Hab ich angefangen. Eine Seite hab ich schon geschrieben, am Ende fahlt noch ein bisschen was. Werde daran noch weiter machen. 
@Amra/Alma: Bitte drüberlesen und vielleicht noch den einen oder anderen Satz einfügen wenn euch was dazu einfällt. 

Kapitel 3)
Ist fertig. 

Kapitel 4) 
David hat gesagt das Kapitel macht er fertig.

Kapitel 5)
Hat der Thomas angefangen (ein paar Zeilen).
@Amra/Alma: Bitte ausarbeiten. Müssen keine 5 Seiten sein, 2-3 reichen auch. Bitte teilt euch ein wer da welche der im Kapitel gestellten Fragen bearbeitet.

LG Stefan

\subsection{24.11.2014 - David}
Hey,

hier ein kleines Update von mir. Hab zu Kapitel 4 ca ne Seite Stichworte und muss das ganze noch in einen gescheiten Text umschreiben, was ich leider gestern (und heute) nicht geschafft habe. Vielleicht habe ich heut, noch vor dem Schlafen gehen, was vorzuzeigen, realistischerweise allerdings erst morgen Abend.

lg David

\subsection{25.11.2014 - Thomas}
ja hallo,
ich weiß, dass ich spät dran bin, aber kann leider erst morgen was zu meinem teil beitragen. bin die woche ein bisschen ins strudeln gekommen ...
 
lg

\subsection{26.11.2014 - Amra, Alma, Stefan}
Hi,
Die Amra hat mich gebeten euch Bescheid zu geben, dass schon einiges zu kapitel 5 fertig ist.
@Thomas vor allem einiges von Berkeley und aus Papers:
- Berkeley research 
- Using brainwaves aus nem guten pdf aus cscjournals
- Das schon in dern 60ern was zum thema ga, 
- 2001 benedicenti, koles und Paranjape "the electriencephalogram as a biometric ......"
- 2008 Palaniappan rR. " two stges biometric authentication method using thought"

Die Alma überarbeitet das noch und fügt es dann am Abend ein. Schau was du noch so findest, z.B über dieses eine Startup, das GoogleGlass mit Brainwaves steuert und zu "critical discussion and description"

Stefan

\subsection{26.11.2014 - Thomas}
ok, wäre es möglich die rohfassung mal zu pushen, damit ich mal den letzten stand hab ...

\subsection{26.11.2014 - Alma}
Hi,
 anbei der derzeitige Stand des Kapitels 5.

Was noch gemacht werden sollte ist:
1 google glas
2 Description and critical discussion of related scientific work (da steht zwar was drinnen, aber ich glaube dass wir uns eher auf die nachteile des derzeitigen produktes konzentrieren sollten - größe des headsets, unpraktisch usw.

\section{Abgabe am 27.11.2014}
\begin{itemize}
  \item Gegenseitiges Lesen der Kapitel, Feedback und kleine Verbesserungen
  \item Bibtex Einträge erstellen von Alma
\end{itemize}

\section{What happened since the last meeting?}
Wir haben 15 Punkte auf die erste Abgabe bekommen!

Feedback: Die Instruktionen von Seite 3 entfernen. ...ist erledigt!


\end{document}
