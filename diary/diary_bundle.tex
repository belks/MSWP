\section{Erstes Treffen - Di 28 Okt
2014}\label{erstes-treffen---di-28-okt-2014}

\begin{itemize}
\item
  \textbf{Anwesende:} alle
\item
  \textbf{Abwesend:} niemand
\item
  \textbf{Ort:} Freihaus, Arbeitsraum gelb
\item
  \textbf{Zeit:} 17:45-20:10
\item
  \textbf{Tagesordnung:} kennenlernen, Rahmen der LVA besprechen,
  Unterlagen sichten und verteilen
\item
  Unterlagen und Vorlagen bereitstellen

  \begin{itemize}
  \itemsep1pt\parskip0pt\parsep0pt
  \item
    Dropboxordner wird erstellt
  \item
    git-Repository wird erstellt
  \item
    alle Vorlagen werden bereitgestellt
  \end{itemize}
\item
  Brainstorming

  \begin{itemize}
  \itemsep1pt\parskip0pt\parsep0pt
  \item
    eRezept (ähnlich eMedikation)
  \item
    Goatsimulator (Gedanken verstehen)
  \item
    App für freie Parkplätze
  \item
    Passwörter über Gehirnwellen eingeben
  \item
    Stalking App (??)
  \item
    Kleidung nach Wetter und Laune (intelligenter Kleiderschrank)
  \item
    Wahlapp (wählt von allein?)
  \item
    Sichere Email Alternative
  \item
    Dioptrin messen / Optiker App
  \item
    Intelligente Taschentücher (Gesundheitszustand)
  \item
    Schnitzelscanner (Essen scannen, Kalorien, Zutaten)
  \item
    Semantic porn/shoes (oder, wer ist gut in Mathe?)
  \item
    OnlineStudentManager
  \end{itemize}
\item
  4 finale Ideen

  \begin{itemize}
  \itemsep1pt\parskip0pt\parsep0pt
  \item
    OnlineStudentManager
  \item
    eRezept
  \item
    BrainCode
  \item
    WhoKnowsWhat
  \end{itemize}
\item
  Eine Mail mit den Vorschlägen wird dem Tutor geschickt
\item
  Planung des nächsten Treffens: Freitag 31.10, 12 Uhr im gleichen Raum.
  Thema. Feedback des Tutors, Ausarbeitung der Grundlagen des Projektes
  und Vorbereitung der Folien
\end{itemize}

\section{Grundlagen des Projektes klären - Fr 31 Okt
2014}\label{grundlagen-des-projektes-kluxe4ren---fr-31-okt-2014}

\begin{itemize}
\item
  \textbf{Anwesende:} alle
\item
  \textbf{Abwesend:} niemand
\item
  \textbf{Ort:} Freihaus, Arbeitsraum gelb
\item
  \textbf{Zeit:} 12:10-14:40
\item
  \textbf{Tagesordnung:} Feedback des Tutors besprechen, Ausarbeitung
  der Grundlagen des Projektes und Vorbereitung der Folien
\item
  Was muss für 6.11 vorbereitet werden?

  \begin{itemize}
  \itemsep1pt\parskip0pt\parsep0pt
  \item
    Präsentation von 7 Minuten
  \item
    Präsentationsfolien in GoogleDocs vorbereitet
  \item
    Projektidee formulieren
  \item
    Warum, Was und Wie kommt auf die Folien
  \item
    lustiges Video über Gehirnwellen-gesteuerte Geräte am Ende zeigen,
    wenn Zeit bleibt
  \end{itemize}
\item
  Diskussion wie das Erkennen der Gedanken funktionieren könnte

  \begin{itemize}
  \itemsep1pt\parskip0pt\parsep0pt
  \item
    mehrere Links zu Artikeln in der Linksammlung hinzugefügt
  \item
    Gehirnwellen können gemessen und Gedanken erkannt werden
  \item
    Ist eine persönliche Analyse relevant? Muss man das Gerät
    trainieren?

    \begin{itemize}
    \itemsep1pt\parskip0pt\parsep0pt
    \item
      mehrere Personen sollen mit verschiedenen Profilen unterstützt
      werden
    \item
      um gute Ergebnisse zu erzielen wird wahrscheinlich ein Training
      notwendig sein (Unterlagen?)
    \end{itemize}
  \end{itemize}
\item
  Diskussion wie das Entsperren/Passworteingabe funktionieren könnte

  \begin{itemize}
  \itemsep1pt\parskip0pt\parsep0pt
  \item
    Es ist sinnvoll mit dem Gerät ein Masterpasswort zu übertragen,
    welches dann einzelne Domänengebundene Passwörter freigeben kann
  \item
    Dies kann z.B. durch eine erweiterung eines im Betriebssystem
    eingebauten Keyrings erreicht werden
  \end{itemize}
\item
  Diskussion über Stromversorgung und Geräteform

  \begin{itemize}
  \itemsep1pt\parskip0pt\parsep0pt
  \item
    Bluetoothheadset ist wegen Akku so groß
  \item
    Hörgeräte sind mittlerweile relativ kein
  \item
    Strominduktion durch Radiowellen (genug Energie?, legal?)
  \item
    Stromversorgung durch Bewegung (Pendel..)
  \item
    lowcost EEG-Sensor
  \item
    Sind maßgeschneiderte Geräte sinnvoll?
  \item
    eventuell in Bluetoothgeräte einbauen
  \item
    an Brillen befestigen (wie Googleglasses)
  \item
    Müssen die Gehirnströme an der Stirn bzw. an mehreren Punkten
    abgenommen werden?
  \end{itemize}
\item
  Diskussion über generelle Sinnhaftigkeit und Alternativen

  \begin{itemize}
  \itemsep1pt\parskip0pt\parsep0pt
  \item
    es ist praktisch
  \item
    es kann von Menschen verwendet werden, die Alternativen wie
    Fingerabdrucksensor oder Retinascans nicht verwenden können
  \item
    Biometrie generell ist nach Kompromittierung permanent unsicher
    (Körperteile lassen sich schwer tauschen\ldots{})
  \item
    Verschmutzung des Scanners nicht tragisch, Handschuhe und Brille
    sind nicht im Weg -\textgreater{} Option für spezielle Situationen
    wie Sport, Winter oder schmutzige/staubige Umgebung, eventuell auch
    unter Wasser (wobei, wer gibt da Passwörter ein..)?
  \end{itemize}
\item
  Diskussion über die Sicherheit der Verbindung

  \begin{itemize}
  \itemsep1pt\parskip0pt\parsep0pt
  \item
    ist es Hackbar?
  \item
    wie sicher ist Bluetoothübertragung? Muss recherchiert werden
  \item
    Sind Gehirnwellenmuster einzigartig? Prinzipiell ja.
  \item
    Sind aktive Gedanken anders als unbewusste? Aktive an Kaffee denken
    vs jemand stellt eine Tasse Kaffee auf den Tisch..
  \end{itemize}
\item
  Es gibt bereits ein Forschungsprojekt zum Thema (2011) -\textgreater{}
  Linksammlung
\item
  Wer präsentiert? Stefan und David
\item
  Brainstorming zu Projektnamen: Brain , Wave, Lock\ldots{}
\item
  Projektname: WaveMeIn
\item
  hm? hm.. hmhm.. hm?
\end{itemize}

\section{Submission 1: Synopsis - Fr 31 Okt
2014}\label{submission-1-synopsis---fr-31-okt-2014}

\begin{itemize}
\item
  \textbf{Anwesende:} Stefan, Alma, Thomas, David
\item
  \textbf{Abwesend:} Amra
\item
  \textbf{Ort:} Freihaus, Arbeitsraum gelb
\item
  \textbf{Zeit:} 12:15-
\item
  \textbf{Tagesordnung:} Submission 1: Synopsis
\item
  Was muss in der Abgabe enthalten sein?

  \begin{itemize}
  \itemsep1pt\parskip0pt\parsep0pt
  \item
    WHY, WHAT, HOW und RESULTS Beschreibung
  \item
    Eine ausgebaute Version der Präsentation
  \end{itemize}
\item
  Das Headset wir ab jetzt Wavy, der Computer oder Handy als Device
  genannt
\item
  Wie soll das Pairen funktionieren?

  \begin{itemize}
  \itemsep1pt\parskip0pt\parsep0pt
  \item
    Ähnlich wie Bluetooth
  \end{itemize}
\item
  Soll das Brainpattern am Wavy oder dem Device ausgewertet werden?

  \begin{itemize}
  \itemsep1pt\parskip0pt\parsep0pt
  \item
    Wenn am Wavy, dann kann man das Pattern einmal trainieren und auf
    mehreren Devices verwenden, solange diese gepairt wurden.
  \item
    Das trainieren ist wahrscheinlich aufwändiger und kann am Device
    erfolgen (Patterns werden übertragen und ausgewertet)
  \end{itemize}
\item
  Wie funktioniert das Entsperren?

  \begin{enumerate}
  \def\labelenumi{\arabic{enumi}.}
  \itemsep1pt\parskip0pt\parsep0pt
  \item
    Wavy ist im passive Modus, dh. hört auf Aktivierungssignale
  \item
    Benutzer aktiviert das Device (z.B. den Entsperrbutton am Handy um
    den Screen für die Mustereingabe anzuzeigen)
  \item
    Device sucht bekannte Wavys und sendet diesen ein
    Authentication-Request
  \item
    Wavy wird aktiviert, messt Hirnwellen, wertet diese aus und
    antwortet ggf. positiv (auch negativ??) an das Device
  \item
    Bei positiver Antwort wird das Device entsperrt
  \end{enumerate}
\item
  Ob sich jemand wirklich dieses Protokoll ansehen wird?
\item
\end{itemize}
